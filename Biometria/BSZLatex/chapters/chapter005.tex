\chapter{Struktura sprawozdania z ćwiczeń laboratoryjnych - Biometryczne systemy zabezpieczeń}
\label{cha:elementySprBSZ}

\section{Instrukcja przygotowania sprawozdania}
Sprawozdanie należy przygotować z wykorzystaniem udostępnionego szablonu \LaTeX. Dokument powinien być zgodny z wymaganym formatowaniem oraz spełniać określone kryteria merytoryczne. Każde sprawozdanie musi być wykonane oddzielnie dla każdego laboratorium.

Sprawozdanie powinno zawierać szczegółowe opracowanie zaproponowanych rozwiązań dla ćwiczeń oraz zadań do samodzielnego rozwiązania. W szczególności należy uwzględnić następujące elementy:
\begin{itemize}
    \item Opis stosowanych algorytmów i metod - krótkie omówienie zastosowanych metod przetwarzania obrazów, w tym np. histogramowej segmentacji, binaryzacji, detekcji cech biometrycznych. Wskazanie ich znaczenia w analizie biometrycznej oraz opis działania.
    \item Opis wykorzystywanego oprogramowania i narzędzi - informacja o środowisku programistycznym (np. MATLAB + Image Processing Toolbox). Opis kluczowych funkcji wykorzystanych w ćwiczeniu. Kod źródłowy programu w postaci listingów umieszczonych zgodnie z wymaganiami dokumentacji technicznej.
    \item Przebieg eksperymentu - szczegółowy opis kolejnych etapów eksperymentu. Przygotowanie danych wejściowych, np. wczytanie obrazu, wstępne przetwarzanie. Realizacja poszczególnych operacji, np. segmentacja, binaryzacja, wykrywanie cech biometrycznych. Analiza wyników pośrednich, np. wykresy histogramów, zwizualizowane obrazy po przekształceniach.
    \item Wyniki i analiza - prezentacja wyników uzyskanych w trakcie eksperymentu. Porównanie różnych metod segmentacji i ich wpływu na jakość analizy biometrycznej. Analiza dokładności uzyskanych wyników oraz identyfikacja potencjalnych błędów.
    \item Wnioski - podsumowanie skuteczności zastosowanych metod. Możliwości poprawy wyników, np. poprzez zastosowanie bardziej zaawansowanych metod segmentacji. Potencjalne zastosowania w rzeczywistych systemach biometrycznych.
    \item Bibliografia - lista wykorzystanych materiałów, w tym podręczników, dokumentacji MATLAB-a, publikacji naukowych i innych źródeł.
\end{itemize}

Dokument należy sporządzić w sposób przejrzysty, stosując numerowane sekcje i podsekcje, zgodnie z zasadami formatowania w \LaTeX.

Praca powinna składać się z poniższych rozdziałów: 
\begin{itemize}
    \item Laboratorium 1
    \item Laboratorium 2
    \item Laboratorium 3
    \item ....
    \item Laboratorium n
    \item Bibliografia
    \item Spis rysunków
    \item Spis tabel
    \item Spis listingów
    \item Oświadczenie studenta o samodzielności pracy 
\end{itemize}





% ********** Koniec **********