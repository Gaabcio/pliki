\chapter{Pierwszy dokument}
\label{cha:pierwszyDokument}

W rozdziale tym przedstawiono podstawowe informacje dotyczące struktury prostych plików \LaTeX~a. Omówiono również metody kompilacji plików z zastosowaniem programów \emph{latex} oraz \emph{pdflatex}.

%---------------------------------------------------------------------------

\section{Struktura dokumentu}
\label{sec:strukturaDokumentu}

Plik \LaTeX~owy jest plikiem tekstowym, który oprócz tekstu zawiera polecenia formatujące ten tekst (analogicznie do języka HTML). Plik składa się z dwóch części:
\begin{enumerate}%[1)]
      \item Preambuły -- określającej klasę dokumentu oraz zawierającej m.in. polecenia dołączającej dodatkowe pakiety;

      \item Części głównej -- zawierającej zasadniczą treść dokumentu.
\end{enumerate}

% ------------------------
\begin{lstlisting}
\documentclass[a4paper,12pt]{article}      % preambuła
\usepackage[polish]{babel}
\usepackage[utf8]{inputenc}
\usepackage[T1]{fontenc}
\usepackage{times}

\begin{document}                           % część główna

\section{Sztuczne życie}

% treść
% ąśężźćńłóĘŚĄŻŹĆŃÓŁ

\end{document}
\end{lstlisting}

Nie ma żadnych przeciwskazań do tworzenia dokumentów w~\LaTeX~u w~języku polskim. Plik źródłowy jest zwykłym plikiem tekstowym i~do jego przygotowania można użyć dowolnego edytora tekstów, a~polskie znaki wprowadzać używając prawego klawisza \texttt{Alt}. Jeżeli po kompilacji dokumentu polskie znaki nie są wyświetlane poprawnie, to na 95\% źle określono sposób kodowania znaków (należy zmienić opcje wykorzystywanych pakietów).
%---------------------------------------------------------------------------


% ------------------------
\section{Kompilacja}
\label{sec:kompilacja}


Załóżmy, że przygotowany przez nas dokument zapisany jest w pliku \texttt{test.tex}. Kolejno wykonane poniższe polecenia (pod warunkiem, że w pierwszym przypadku nie wykryto błędów i kompilacja zakończyła się sukcesem) pozwalają uzyskać nasz dokument w formacie pdf:
\begin{lstlisting}
latex test.tex
dvips test.dvi -o test.ps
ps2pdf test.ps
\end{lstlisting}
%
lub za pomocą PDF\LaTeX:
\begin{lstlisting}
pdflatex test.tex
\end{lstlisting}

Przy pierwszej kompilacji po zmiane tekstu, dodaniu nowych etykiet itp., \LaTeX~tworzy sobie spis rozdziałów, obrazków, tabel itp., a dopiero przy następnej kompilacji korzysta z tych informacji.

W pierwszym przypadku rysunki powinny być przygotowane w~formacie eps, a~w~drugim w~formacie pdf. Ponadto, jeżeli używamy polecenia \texttt{pdflatex test.tex} można wstawiać grafikę bitową (np. w formacie jpg).



%---------------------------------------------------------------------------

\section{Narzędzia}
\label{sec:narzedzia}

Istnieje wiele narzędzi do pisania w \LaTeX, które mogą pomóc w tworzeniu dokumentów i zarządzaniu nimi. Oto lista popularnych narzędzi i edytorów \LaTeX, które są dostępne zarówno w wersji online, jak i offline.

\noindent \textbf{Narzędzia online do pisania w~\LaTeX.}
\begin{itemize}
      \item Overleaf -- jeden z najpopularniejszych edytorów \LaTeX~online. Umożliwia współpracę z innymi użytkownikami w czasie rzeczywistym oraz zapewnia automatyczną kompilację dokumentów. Overleaf oferuje szereg gotowych szablonów (np. prace dyplomowe, CV, artykuły naukowe). Wsród funkcji można wyróżnić m.in.: kompilacja \LaTeX~w czasie rzeczywistym, wsparcie dla XeLaTeX, LuaLaTeX, BibTeX, Biber, współpraca z wieloma użytkownikami, gotowe szablony oraz integracja z GitHub. Szczegółowe informacje dostępne są pod adresem: \url{https://www.overleaf.com/}.
      \item Papeeria -- edytor \LaTeX~online podobny do Overleaf. Umożliwia współpracę w czasie rzeczywistym i ma integrację z GitHub. Oferuje bezpłatną wersję z podstawowymi funkcjami. Funkcje: wsparcie dla współpracy online, kompilacja \LaTeX~w chmurze. Szczegółowe informacje dostępne są pod adresem: \url{https://www.papeeria.com/}.
\end{itemize}

\noindent \textbf{Edytory offline do pisania w LaTeX.}

\begin{itemize}
      \item Visual Studio Code (VSC) -- to potężne narzędzie do pisania dokumentów w \LaTeX, szczególnie po zainstalowaniu rozszerzenia \LaTeX~Workshop. Zapewnia ono pełne wsparcie dla kompilacji, podglądu PDF, inteligentnego uzupełniania składni oraz współpracy z narzędziami bibliograficznymi jak BibTeX i Biber. Dzięki integracji z systemem zarządzania plikami i narzędziami do wersjonowania (jak Git), VS Code staje się bardzo elastycznym narzędziem do pisania profesjonalnych dokumentów, artykułów naukowych i prac dyplomowych w \LaTeX~. Szczegółowe informacje odnośnie konfiguracji środowiska do pisania LaTeX w VSC można znelźć a instrukcji \textit{"How to write LaTeX documents using Visual Studio Code"} pod adresem \url{https://www.geekering.com/programming-languages/filipesalgueiro/how-to-write-latex-documents-using-visual-studio-code/}
      \item TeXstudio -- to zintegrowane środowisko do tworzenia dokumentów \LaTeX~. Celem jest uczynienie pisania w \LaTeX~u tak łatwym i wygodnym, jak to tylko możliwe. Dlatego TeXstudio ma wiele funkcji, takich jak podświetlanie składni, zintegrowana przeglądarka, sprawdzanie referencji i różni asystenci. Więcej szczegółów można znaleźć w funkcjach. Szczegółowe informacje dostępne są pod adresem: \url{https://www.texstudio.org/}
\end{itemize}

\noindent \textbf{Dodatkowe narzędzia wspierające pisanie w LaTeX.}
\begin{itemize}
      \item Mendeley -- popularne narzędzie do zarządzania bibliografią i organizacji dokumentów naukowych. Obsługuje format BibTeX i pozwala na łatwą integrację z LaTeX. Szczegółowe informacje dostępne są pod adresem: \url{https://www.mendeley.com}.
      \item JabRef -- darmowe narzędzie do zarządzania bibliografią i referencjami, które wspiera format BibTeX i BibLaTeX. Ułatwia organizowanie źródeł bibliograficznych i integrację z LaTeX. Zobacz: \url{https://www.jabref.org/}.
      \item Zotero -- kolejne narzędzie do zarządzania referencjami naukowymi, które wspiera BibTeX i BibLaTeX. Zotero jest popularnym wyborem wśród studentów i badaczy. Szczegółowe informacje dostępne są pod adresem: \url{https://www.zotero.org}.
\end{itemize}

Wśród środowisk online wyrżnić można: Overleaf, Papeeria i Authorea oferują wygodną współpracę i szybki dostęp do narzędzi LaTeX z dowolnego miejsca. Środowiska offline: TeXmaker, TeXworks, Kile i LyX to doskonałe edytory dla pracy lokalnej z rozbudowanymi funkcjami. Dodatkowe narzędzia: JabRef, Mendeley i Zotero ułatwiają zarządzanie bibliografią i cytowaniami w LaTeX. Wybór odpowiedniego narzędzia zależy od Twoich potrzeb, czy preferujesz pracę lokalną, czy online, a także jak bardzo zaawansowane funkcje są Ci potrzebne.

%---------------------------------------------------------------------------

\section{Przygotowanie dokumentu}
\label{sec:przygotowanieDokumentu}

Plik źródłowy \LaTeX a jest zwykłym plikiem tekstowym. Przygotowując plik
źródłowy warto wiedzieć o kilku szczegółach:

\begin{itemize}
      \item
            Poszczególne słowa oddzielamy spacjami, przy czym ilość spacji nie ma znaczenia.
            Po kompilacji wielokrotne spacje i tak będą wyglądały jak pojedyncza spacja.
            Aby uzyskać {\em twardą spację}, zamiast znaku spacji należy użyć znaku {\em
                        tyldy}.

      \item
            Znakiem końca akapitu jest pusta linia (ilość pusty linii nie ma znaczenia), a
            nie znaki przejścia do nowej linii.

      \item
            \LaTeX~sam formatuje tekst. \textbf{Nie starajmy się go poprawiać}, chyba, że
            naprawdę wiemy co robimy.
\end{itemize}


\noindent Przydatna pomoc do pracy w \LaTeX:
\begin{itemize}
      \item Dokumentacja dostęna w Overleaf \url{https://www.overleaf.com/learn}.
      \item Edytor równań dla matematyki online\url{https://editor.codecogs.com/}.
      \item  Edytor/generator tabel \url{https://www.tablesgenerator.com/}.
\end{itemize}


\section{Sprawdzanie pisowni w VSC}

Visual Studio Code można sprawdzać pisownię w dokumentach \LaTeX~, w tym także w języku polskim. W tym celu należy zainstalować poniższe rozszerzenia.
\begin{enumerate}
      \item Zainstalowanie rozszerzenia "Code Spell Checker" - To rozszerzenie (streetsidesoftware.code-spell-checker) umożliwia sprawdzanie pisowni w różnych językach. Poniżej kroki jak dodać obsługę języka polskiego:
      \begin{itemize}
            \item Otwórz Visual Studio Code.
            \item Przejdź do Extensions (Ctrl + Shift + X).
            \item Wyszukaj i zainstaluj Code Spell Checker.
            \item Następnie zainstaluj rozszerzenie Polish - Code Spell Checker (streetsidesoftware.code-spell-checker-polish). Po instalacji sprawdzanie pisowni dla języka polskiego powinno działać
      \end{itemize}
      \item Dodanie języka polskiego do konfiguracji - jeśli chcesz, aby sprawdzanie pisowni działało domyślnie w języku polskim, dodaj wpis do ustawień:
      \begin{itemize}
            \item Otwórz Ustawienia (Ctrl + ,).
            \item Wyszukaj "cSpell.language".
            \item Kliknij "Edit in settings.json".
            \item Dodaj lub edytuj ustawienie: "cSpell.language": "pl,en". Możesz też dodać pl do listy obsługiwanych języków, jeśli już masz tam en.
      \end{itemize} 
      \item Sprawdzanie pisowni w czasie rzeczywistym  - Po instalacji rozszerzenia błędnie napisane słowa będą podkreślane na czerwono. Możesz kliknąć na podkreślone słowo i wybrać poprawną sugestię.
\end{enumerate}

